\documentclass{article}%
\usepackage[T1]{fontenc}%
\usepackage[utf8]{inputenc}%
\usepackage{lmodern}%
\usepackage{textcomp}%
\usepackage{lastpage}%
\usepackage[tmargin=1.5cm,lmargin=2.5cm]{geometry}%
\usepackage{graphicx}%
\usepackage{subcaption}%
%
%
%
\begin{document}%
\normalsize%
\section*{CLASSIFICATION REPORT}%
\label{sec:CLASSIFICATIONREPORT}%
\subsection*{Architecture of the Neural Network}%
\label{subsec:ArchitectureoftheNeuralNetwork}%


\begin{figure}[h!]%
\centering%
\includegraphics[width=200px]{images/model.pdf}%
\end{figure}

%
\subsection*{TP/FP/TN/FN Test Set Examples}%
\label{subsec:TP/FP/TN/FNTestSetExamples}%
TP {-} true positives, TN {-} true negatives, FP {-} false positives, FN {-} False negaties%


\begin{figure}[h!]%
\centering%
\begin{subfigure}[c]{0.33\linewidth}%
\includegraphics[width=0.95\linewidth]{images/examples1.pdf}%
\caption{Examples of TP}%
\end{subfigure}%
\begin{subfigure}[c]{0.33\linewidth}%
\includegraphics[width=0.95\linewidth]{images/examples2.pdf}%
\caption{Examples of FP}%
\end{subfigure}%
\end{figure}

%


\begin{figure}[h!]%
\centering%
\begin{subfigure}[c]{0.33\linewidth}%
\includegraphics[width=0.95\linewidth]{images/examples1.pdf}%
\caption{Examples of TN}%
\end{subfigure}%
\begin{subfigure}[c]{0.33\linewidth}%
\includegraphics[width=0.95\linewidth]{images/examples2.pdf}%
\caption{Examples of FN}%
\end{subfigure}%
\end{figure}

%
\subsection*{Comparison of True Labes and Output Values for Test Set:}%
\label{subsec:ComparisonofTrueLabesandOutputValuesforTestSet}%


\begin{figure}[h!]%
\centering%
\includegraphics[width=200px]{images/true_pred.pdf}%
\end{figure}

%
\subsection*{Training and Validation Loss and Accuracy:}%
\label{subsec:TrainingandValidationLossandAccuracy}%


\begin{figure}[h!]%
\centering%
\includegraphics[width=260px]{images/loss_acc.pdf}%
\end{figure}

%
\subsection*{Test Set Precission and Recall:}%
\label{subsec:TestSetPrecissionandRecall}%


\begin{figure}[h!]%
\centering%
\includegraphics[width=260px]{images/prec_recall.pdf}%
\end{figure}

%
\subsection*{Test Set Confusion Matrix}%
\label{subsec:TestSetConfusionMatrix}%


\begin{figure}[h!]%
\centering%
\includegraphics[width=210px]{images/conf.pdf}%
\end{figure}

%
\subsection*{Classification Scoring for Test Set}%
\label{subsec:ClassificationScoringforTestSet}%
TP {-} true positives, TN {-} true negatives, FP {-} false positives, FN {-} False negaties \newline%
\newline%
%
The performance of a classifier can be described by:\newline%
%
\textbf{Accuracy }%
 {-} (TP+TN)/(TP+TN+FP+FN) \newline%
%
\textbf{Precision }%
 (Purity, Positive Predictive Value) {-} TP/(TP+FP) \newline%
%
\textbf{Recall }%
 (Completeness, True Positive Rate {-} TP/(TP+FN) \newline%
 %
\textbf{F1 Score }%
 = 2 (Precision * Recall)/(Precision + Recall).\newline%
%
\textbf{Brier Score }%
 {-} mean squared error (MSE) between predicted probabilities (between 0 and 1) and the expected values (0 or 1). Brier score summarizes the magnitude of the forecasting error and takes a value between 0 and 1 (with better models having score close to 0).\newline%
\newline%
%
\begin{tabular}{|l|l|}%
\hline%
\textbf{Metric}&\textbf{Score}\\%
\hline%
Accuracy&0.75\\%
Precision&0.8\\%
Recall&0.85\\%
F1 Score&0.01\\%
Brier Score&0.1\\%
\hline%
\end{tabular}%
\newline%
\newline%

%
\subsection*{Reciever Operating Characteristic (ROC) and Area Under the Curve (AUC) for Test Set}%
\label{subsec:RecieverOperatingCharacteristic(ROC)andAreaUndertheCurve(AUC)forTestSet}%
The ROC curve graphically shows the trade{-}off between between true{-}positive rate and false{-}positive rate.The AUC summarizes the ROC curve {-} where the AUC is close to unity, classification is successful, while an AUC of 0.5 indicates the model is performs as well as a random guess.%


\begin{figure}[h!]%
\centering%
\includegraphics[width=220px]{images/ROC.pdf}%
\hspace*{2cm}%
\begin{large}%
AUC = 0.75\newline%
\newline%
\newline%
\newline%
\newline%
\newline%
\newline%
\newline%
%
\end{large}%
\end{figure}

%
\subsection*{Histogram of the Output Probabilities for Test Set}%
\label{subsec:HistogramoftheOutputProbabilitiesforTestSet}%


\begin{figure}[h!]%
\centering%
\includegraphics[width=230px]{images/histogram.pdf}%
\end{figure}

%
\subsection*{2D Histogram of the Output vs One Object Feature for Test Set}%
\label{subsec:2DHistogramoftheOutputvsOneObjectFeatureforTestSet}%


\begin{figure}[h!]%
\centering%
\includegraphics[width=230px]{images/2Dhistogram.pdf}%
\end{figure}

%
\end{document}